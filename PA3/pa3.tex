\documentclass[11pt]{article}
\author{Hasan Tahir Abbas}
\usepackage{graphicx}
%\usepackage{epstopdf}
\usepackage{pgfplots}
\usepackage{subfig}
\usepackage{tikz}
\usepackage{times}
%\usepackage{tikzpicture}
\usepackage{standalone}
\renewcommand\footnotemark{}
\renewcommand\footnoterule{}
%
%
%%%%%%%%%%%%%%%%%%%%%%%%%%%%%%%%%
\hoffset        2mm 
\voffset        -1.5in
\oddsidemargin  0mm
\topmargin      0.5in
\textwidth      6in
\textheight     9in

\usepackage{graphicx}
\usepackage{fancyhdr}
\pagestyle{fancy}
\fancyhf{}
\fancyhead[R]{\thepage}
%%%%%%%%%%%%%%%%%%%%%%%%%%%%%%%%%
\begin{document}

\begin{titlepage}

\vspace*{55mm}
\begin{center}
{\huge MATH 609}\\[1cm]
{\em \huge Homework \#5}\\[70mm]
{\large Fall, 2015} \\[15mm]
\end{center}

\begin{flushright}
{\LARGE Hasan Tahir Abbas}
\end{flushright}

\vfill

\end{titlepage}
%
%
%
%
\newpage
\section{Specifications}
\subsection{Tridiagonal System of Equations }

In the first case, the numerical solution is compared with the exact solution when the $k$ parameter is approximated by a linear function. We see that the performance of Conjugate-Gradient (CG) method is much faster than the Steepest-Descent (SD) method. In the second part, numerical solution is computed when the $k$ parameter is a piece-wise function \footnote{$\varsigma$: Convergence not achieved within 100000 iterations}.


\begin{table}[!hbt]
\begin{center}
{\def\arraystretch{.95}
\begin{tabular}{|c|c|c|c|}
\hline
Method  & $n = 19$  &  $n = 39$  &  $n = 79$\\
\hline
CG & 20 & 42& 89\\ 
\hline
SD & 2430 & 9761 & 38467\\ 
\hline
\end{tabular}}
\end{center}
\caption{Iterations for Convergence required in Example 1, part a}
\end{table}
%
%
\begin{table}[!hbt]
\begin{center}
{\def\arraystretch{.95}
\begin{tabular}{|c|c|c|c|}
\hline
$n \downarrow$ & $K = 2$  &  $K = 100$  &  $K = 1000$\\
\hline
19 & 21 & 30& 37\\ 
\hline
39 & 46 & 84& 107\\ 
\hline
79 & 96 & 21& 311\\ 
\hline
\end{tabular}}
\end{center}
\caption{Iterations for Convergence required in Example 1, part b for CG method}
\end{table}
%
%
\begin{table}[!hbt]
\begin{center}
{\def\arraystretch{.95}
\begin{tabular}{|c|c|c|c|}
\hline
$n \downarrow$ & $K = 2$  &  $K = 100$  &  $K = 1000$\\
\hline
19 & 2715 & 43676 & $\varsigma$ \\ 
\hline
39 & 10553 & $\varsigma$ & $\varsigma$\\ 
\hline
79 & 40509 & $\varsigma$ & $\varsigma$\\ 
\hline
\end{tabular}}
\end{center}
\caption{Iterations for Convergence required in Example 1, part b for SD method  }
\end{table}

%%%%%%%%%%%%%%%%%%%%%%%%%%%%%%%
%%%%%%%%%%%%%%%%%%%%%%%%%%%%%%%
\subsection{Approximate Solution of 2D Elliptic Equation}

The numerical solution of a two-dimensional elliptic equation is computed by applying the given boundary conditions.


\begin{table}[!hbt]
\begin{center}
{\def\arraystretch{.95}
\begin{tabular}{|c|c|c|c|}
\hline
Method  & $n = 8$  &  $n = 16$  &  $n = 32$\\
\hline
CG & 6 & 27&67\\ 
\hline
SD & 254 & 1291 & 5639\\ 
\hline
\end{tabular}}
\end{center}
\caption{Iterations for Convergence required in Example 2}
\end{table}
%
%
%%%%%%%%%%%%%%%%%%%%%%%%%%%%%%%
%%%%%%%%%%%%%%%%%%%%%%%%%%%%%%%
\subsection{Numerical Solution of Trough Potential}

The numerical solution of an electric potential in a trough is computed. The top boundary has a voltage boundary condition of 100 volts and the rest of the boundaries are perfect electric conductors (PEC) having zero potential. The region is assumed square.


\begin{table}[!hbt]
\begin{center}
{\def\arraystretch{.95}
\begin{tabular}{|c|c|c|c|}
\hline
Method  & $n = 8$  &  $n = 16$  &  $n = 32$\\
\hline
CG & 13 & 48&165\\ 
\hline
SD & 236 & 1128 &4689\\ 
\hline
\end{tabular}}
\end{center}
\caption{Iterations for Convergence required in Example 2}
\end{table}
%
%




%
%
%
%
\newpage
\begin{figure}
     \centering
    		 \subfloat[][]{ \includestandalone[scale=.55]{math609_pa3_comp_example_1_n_19_CG_part_a}\label{n_19_CG_part_a}}
     		 \subfloat[][]{ \includestandalone[scale=.55]{math609_pa3_comp_example_1_n_19_SD_part_a}\label{n_19_SD_part_a}}
\\
    		 \subfloat[][]{ \includestandalone[scale=.55]{math609_pa3_comp_example_1_n_39_CG_part_a}\label{n_39_CG_part_a}}
     		 \subfloat[][]{ \includestandalone[scale=.55]{math609_pa3_comp_example_1_n_39_SD_part_a}\label{n_39_SD_part_a}}
\\
    		 \subfloat[][]{ \includestandalone[scale=.55]{math609_pa3_comp_example_1_n_79_CG_part_a}\label{n_79_CG_part_a}}
     		 \subfloat[][]{ \includestandalone[scale=.55]{math609_pa3_comp_example_1_n_79_SD_part_a}\label{n_79_SD_part_a}}
     \centering
     \caption{Solution of Tridiagonal System with linear function $k(t)$ (a)-(b) $n = 19$ (c)-(d) $n = 39$ (e)-(f) $n = 79$ with CG and SD method as shown}
     \label{Tradiagonal}
\end{figure}
%
%
\begin{figure}
     \centering
    		 \subfloat[][]{ \includestandalone[scale=.55]{math609_pa3_comp_example_1_n_19_K_2_CG_part_b}\label{n_19_CG_K_2_part_b}}
     		 \subfloat[][]{ \includestandalone[scale=.55]{math609_pa3_comp_example_1_n_19_K_2_SD_part_b}\label{n_19_SD_K_2_part_b}}
\\
    		 \subfloat[][]{ \includestandalone[scale=.55]{math609_pa3_comp_example_1_n_19_K_100_CG_part_b}\label{n_19_CG_K_100_part_b}}
     		 \subfloat[][]{ \includestandalone[scale=.55]{math609_pa3_comp_example_1_n_19_K_100_SD_part_b}\label{n_19_SD_K_100_part_b}}
\\
    		 \subfloat[][]{ \includestandalone[scale=.55]{math609_pa3_comp_example_1_n_19_K_1000_CG_part_b}\label{n_19_CG_K_1000_part_b}}
     		 \subfloat[][]{ \includestandalone[scale=.55]{math609_pa3_comp_example_1_n_19_K_1000_SD_part_b}\label{n_19_SD_K_1000_part_b}}
     \centering
     \caption{Solution of Tridiagonal System with piecewise function $k(t)$  and $n = 19$ (a)-(b) $K = 2$ (c)-(d) $K = 100$ (e)-(f) $K = 1000$ with CG and SD methods as shown}
     \label{Tradiagonal_linear}
\end{figure}
%
%
\begin{figure}
     \centering
    		 \subfloat[][]{ \includestandalone[scale=.55]{math609_pa3_comp_example_1_n_39_K_2_CG_part_b}\label{n_39_CG_K_2_part_b}}
     		 \subfloat[][]{ \includestandalone[scale=.55]{math609_pa3_comp_example_1_n_39_K_2_SD_part_b}\label{n_39_SD_K_2_part_b}}
\\
    		 \subfloat[][]{ \includestandalone[scale=.55]{math609_pa3_comp_example_1_n_39_K_100_CG_part_b}\label{n_39_CG_K_100_part_b}}
     		 \subfloat[][]{ \includestandalone[scale=.55]{math609_pa3_comp_example_1_n_39_K_100_SD_part_b}\label{n_39_SD_K_100_part_b}}
\\
    		 \subfloat[][]{ \includestandalone[scale=.55]{math609_pa3_comp_example_1_n_39_K_1000_CG_part_b}\label{n_39_CG_K_1000_part_b}}
     		 \subfloat[][]{ \includestandalone[scale=.55]{math609_pa3_comp_example_1_n_39_K_1000_SD_part_b}\label{n_39_SD_K_1000_part_b}}
     \centering
     \caption{Solution of Tridiagonal System with piecewise function $k(t)$  and $n = 39$ (a)-(b) $K = 2$ (c)-(d) $K = 100$ (e)-(f) $K = 1000$ with CG and SD methods as shown}
     \label{Tradiagonal_linear}
\end{figure}
%
%
\newpage

\begin{figure}
     \centering
    		 \subfloat[][]{ \includestandalone[scale=.55]{math609_pa3_comp_example_1_n_79_K_2_CG_part_b}\label{n_79_CG_K_2_part_b}}
     		 \subfloat[][]{ \includestandalone[scale=.55]{math609_pa3_comp_example_1_n_79_K_2_SD_part_b}\label{n_79_SD_K_2_part_b}}
\\
    		 \subfloat[][]{ \includestandalone[scale=.55]{math609_pa3_comp_example_1_n_79_K_100_CG_part_b}\label{n_79_CG_K_100_part_b}}
     		 \subfloat[][]{ \includestandalone[scale=.55]{math609_pa3_comp_example_1_n_79_K_100_SD_part_b}\label{n_79_SD_K_100_part_b}}
\\
    		 \subfloat[][]{ \includestandalone[scale=.55]{math609_pa3_comp_example_1_n_79_K_1000_CG_part_b}\label{n_79_CG_K_1000_part_b}}
     		 \subfloat[][]{ \includestandalone[scale=.55]{math609_pa3_comp_example_1_n_79_K_1000_SD_part_b}\label{n_79_SD_K_1000_part_b}}
     \centering
     \caption{Solution of Tridiagonal System with piecewise function $k(t)$  and $n = 79$ (a)-(b) $K = 2$ (c)-(d) $K = 100$ (e)-(f) $K = 1000$ with CG and SD methods as shown}
     \label{Tradiagonal_piecewise}
\end{figure}
%
%
\newpage

\begin{figure}
     \centering
    		 \subfloat[][]{ \includestandalone[scale=.55]{math609_pa3_comp_example_2_8_n_CG}\label{8_n_CG}}
    		 \subfloat[][]{ \includestandalone[scale=.55]{math609_pa3_comp_example_2_8_n_SD}\label{8_n_SD}}
\\
    		 \subfloat[][]{ \includestandalone[scale=.55]{math609_pa3_comp_example_2_16_n_CG}\label{16_n_CG}}
    		 \subfloat[][]{ \includestandalone[scale=.55]{math609_pa3_comp_example_2_16_n_SD}\label{16_n_SD}}
\\
    		 \subfloat[][]{ \includestandalone[scale=.55]{math609_pa3_comp_example_2_32_n_CG}\label{32_n_CG}}
    		 \subfloat[][]{ \includestandalone[scale=.55]{math609_pa3_comp_example_2_32_n_SD}\label{32_n_SD}}
     \centering
     \caption{Approximate Solution of 2D Elliptic Equation $-\Delta u +u = 1$  with $\Omega = (0 ,1)\times(0 ,1)$ and $u(\Omega) = 0$ (a)-(b) $n = 8$ (c)-(d) $n = 16$ (e)-(f) $n = 32$ with CG and SD methods as shown}
     \label{Tradiagonal}
\end{figure}
%
%
\newpage

\begin{figure}
     \centering
    		 \subfloat[][]{ \includestandalone[scale=.55]{math609_pa3_comp_example_3_8_n_CG}\label{3_8_n_CG}}
    		 \subfloat[][]{ \includestandalone[scale=.55]{math609_pa3_comp_example_3_8_n_SD}\label{3_8_n_SD}}
\\
    		 \subfloat[][]{ \includestandalone[scale=.55]{math609_pa3_comp_example_3_16_n_CG}\label{3_16_n_CG}}
    		 \subfloat[][]{ \includestandalone[scale=.55]{math609_pa3_comp_example_3_16_n_SD}\label{3_16_n_SD}}
\\
    		 \subfloat[][]{ \includestandalone[scale=.55]{math609_pa3_comp_example_3_32_n_CG}\label{3_32_n_CG}}
    		 \subfloat[][]{ \includestandalone[scale=.55]{math609_pa3_comp_example_3_32_n_SD}\label{3_32_n_SD}}
\\    		 
     \centering
     \caption{Approximate Electrical Potential Solution of 2D Poisson's Equation $-\Delta \Phi +\Phi = V$  with three PEC boundaries $\Phi(\Omega) = 0$ and 100 volts plate at the top and $\Phi(\chi) = 100$ (a)-(b) $n = 32$ with CG and SD methods as shown}
     \label{Trough}
\end{figure}
\end{document}








