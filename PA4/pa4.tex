\documentclass[11pt]{article}
\author{Hasan Tahir Abbas}
\usepackage{graphicx}
%\usepackage{epstopdf}
\usepackage{pgfplots}
\usepackage{subfig}
\usepackage{tikz}
\usepackage{times}
\usepackage{booktabs}
\usepackage{amsmath}
%\usepackage{tikzpicture}
\usepackage{standalone}
\renewcommand\footnotemark{}
\renewcommand\footnoterule{}
%
%
%%%%%%%%%%%%%%%%%%%%%%%%%%%%%%%%%
\hoffset        2mm 
\voffset        -15mm
\oddsidemargin  0mm
\topmargin      0.5in
\textwidth      6in
\textheight     9in

\usepackage{graphicx}
\usepackage{fancyhdr}
\pagestyle{fancy}
\fancyhf{}
\fancyhead[R]{\thepage}
%%%%%%%%%%%%%%%%%%%%%%%%%%%%%%%%%
\begin{document}

\begin{titlepage}

\vspace*{55mm}
\begin{center}
{\huge MATH 609}\\[1cm]
{\em \huge Programming Assignment \#4}\\[70mm]
{\large Fall, 2015} \\[15mm]
\end{center}

\begin{flushright}
{\LARGE Hasan Tahir Abbas}
\end{flushright}

\vfill

\end{titlepage}
%
%
%
%
\newpage
\section{Specifications}

This programming assignment is implemented through two pieces of algorithms \textit{spline3\_coeff} and \textit{spline3\_eval} as presented in \textit{Numerical Mathematics and Computing} by Kincaid and Cheney. The first code, \textit{spline3\_coeff} accepts a data file and finds the unknown parameters $z_i$ based on the conditions provided at the ends. The second code, \textit{spline3\_eval} computes the spline along with the first and second derivatives of the spline function.

\begin{table}[htbp]
  \centering
\caption{Car Traveling on a Straight road}
    \begin{tabular}{rr}
    \toprule
    Time & Distance\\
    \midrule
    0     & 0 \\
    3     & 225 \\
    5     & 385 \\
    8     & 623 \\
    13    & 933 \\
    \bottomrule
    \end{tabular}%
\label{tab:Car}
\end{table}%


\subsection{Traveling Car Example with free ends}

This example is solved without any modifications to the core algorithms. With the data given in Table \ref{tab:Car}, the position of the car is shown in the Fig. \ref{fig:Car_free}. The speed at $t = 10$ sec is $62.44$ ft/s.

\begin{figure}
     \centering
    		 {\includestandalone[scale=1]{math609_pa4_example_1}}
     \caption{Cubic Spline Interpolation of a traveling car with free ends}
     \label{fig:Car_free}
\end{figure}

\subsection{Traveling Car Example with fixed ends}

Since the end conditions are specified in the case of fixed ends, $z_0$ and $z_n$ are calculated through \ref{eq:fixed_z0} and \ref{eq:fixed_zn}. With the data given in \ref{tab:Car}, the position of the car is shown in the Fig. \ref{fig:Car_free}. The speed at $t = 10$ sec is $64.17$ ft/s.

\begin{figure}
     \centering
    		 {\includestandalone[scale=1]{math609_pa4_example_2}}
     \caption{Cubic Spline Interpolation of a traveling car with fixed ends}
     \label{fig:Car_fixed}
\end{figure}

\begin{equation}
z_0 = 6 \times \frac{\alpha_0 + z_1\times h_0/6 + y_0/h_0 - y_1/h_0}{4h_0}
\label{eq:fixed_z0}
\end{equation}

\begin{equation}
z_n =\left ( (\alpha_n - z_{n-1})\times h_{n-1}/6 - \frac{y_{n}- y_{n-1}}{h_{n-1}}\right )\times 3/h_{n-1}
\label{eq:fixed_zn}
\end{equation}

%%%%%%%%%%%%%%%%%%%%%%%%%%%%%%%
%%%%%%%%%%%%%%%%%%%%%%%%%%%%%%%
\subsection{Approximation of US Population}

This example is solved without any modifications to the core algorithms. The required numbers are tabulated in Table \ref{tab:Pop} and the spline curve is shown in Fig. \ref{fig:Pop}.

\begin{figure}
     \centering
    		 {\includestandalone[scale=1]{math609_pa4_example_3}}
     \caption{Cubic Spline Interpolation of US Population}
     \label{fig:Pop}
\end{figure}

\begin{table}[htbp]
  \centering
  \caption{US Population}
    \begin{tabular}{rr}
    \toprule
    Year & Population\\
    \midrule
    1930  & 123203 \\
    1940  & 131669 \\
    1950  & 150697 \\
    1960  & 179323 \\
    1965  & 191847 \\
    1970  & 203212 \\
    1975  & 214773 \\
    1980  & 226505 \\
    1985  & 238123 \\
    1990  & 249643 \\
    \bottomrule
    \end{tabular}%
  \label{tab:Pop}%
\end{table}%
%
%
%%%%%%%%%%%%%%%%%%%%%%%%%%%%%%%
%%%%%%%%%%%%%%%%%%%%%%%%%%%%%%%
\subsection{Parametric Splines}

In the first example, we have two trigonometric functions, $x(t) = cos(t)$ and  $y(t) = sin(t)$  with $0 <= t <=\pi$. Two plots of the parametric spline are shown for different number of knots are shown in Fig. \ref{fig:trig}. The errors are tabulated in Table \ref{tab:error}.

\begin{figure}
     \centering
    		 \subfloat[][]{ \includestandalone[scale=1]{math609_pa4_example_3_1__n_11}\label{n_11}}\\
		 \subfloat[][]{ \includestandalone[scale=1]{math609_pa4_example_3_1__n_21}\label{n_21}}
     \caption{Parametric Spline for trigonometric functions, $x(t) = cos(t)$ and  $y(t) = sin(t)$  with $0 <= t <=\pi$ (a) $n = 11$ (b) $n = 21$}
     \label{fig:trig}
\end{figure}
%
%
\begin{table}[htbp]
  \centering
  \caption{ Error $\max _{t\in \left[ 0,\pi \right] }\left(  \left| x^{(l)}(t) \right| - \left| S^{(l)}_x (t) \right|   +   \left| y^{(l)}(t) \right| - \left| S^{(l)}_y (t) \right|\right)$ }
    \begin{tabular}{rr}
    \toprule
    \textit{l} & Error\\
    \midrule
    0     & 0.0061 \\
    1     & 0.0791 \\
    2     & 0.6362 \\
    \bottomrule
    \end{tabular}%
  \label{tab:error}%
\end{table}%

%%%%%%%%%%%%%%%%%%%%%%%%%%%%%%%
%%%%%%%%%%%%%%%%%%%%%%%%%%%%%%%
\subsection{Parametric Splines Curve Approximation}

In this example, a curve is first hand-drawn and the co-ordinates of a number of nodes are noted. The list of the co-ordinates is then passed to the code to generate Fig. \ref{fig:curve}. 

\begin{figure}
     \centering
    		 {\includestandalone[scale=1]{math609_pa4_example_3_2}}
     \caption{Parametric Cubic Spline for a hand-drawn curve with n = 11}
     \label{fig:curve}
\end{figure}

%
%
%
%
%\newpage
%\begin{figure}
%     \centering
%    		 \subfloat[][]{ \includestandalone[scale=.55]{math609_pa3_comp_example_1_n_19_CG_part_a}\label{n_19_CG_part_a}}
%     \caption{Solution of Tridiagonal System with linear function $k(t)$ (a)-(b) $n = 19$ (c)-(d) $n = 39$ (e)-(f) $n = 79$ with CG and SD method as shown}
%     \label{Tradiagonal}
%\end{figure}
%
%


\end{document}








