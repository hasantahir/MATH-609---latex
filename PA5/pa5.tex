\documentclass[11pt]{article}
\author{Hasan Tahir Abbas}
\usepackage{graphicx}
%\usepackage{epstopdf}
\usepackage{pgfplots}
\usepackage{subfig}
\usepackage{tikz}
\usepackage{times}
\usepackage{booktabs}
\usepackage{amsmath}
\usepackage{chemformula}
%\usepackage{tikzpicture}
\usepackage{standalone}
\renewcommand\footnotemark{}
\renewcommand\footnoterule{}
%
%
%%%%%%%%%%%%%%%%%%%%%%%%%%%%%%%%%
\hoffset        2mm 
\voffset        -15mm
\oddsidemargin  0mm
\topmargin      0.5in
\textwidth      6in
\textheight     9in

\usepackage{graphicx}
\usepackage{fancyhdr}
\pagestyle{fancy}
\fancyhf{}
\fancyhead[R]{\thepage}
%%%%%%%%%%%%%%%%%%%%%%%%%%%%%%%%%
\begin{document}

\begin{titlepage}

\vspace*{55mm}
\begin{center}
{\huge MATH 609}\\[1cm]
{\em \huge Programming Assignment \#5}\\[70mm]
{\large Fall, 2015} \\[15mm]
\end{center}

\begin{flushright}
{\LARGE Hasan Tahir Abbas}
\end{flushright}

\vfill

\end{titlepage}
%
%
%
%
\newpage
\section{Specifications}

This programming assignment follows algorithms \textit{RK45} and \textit{RK45\_Adaptive} as presented in \textit{Numerical Mathematics and Computing} by Kincaid and Cheney. 

%%%%%%%%%%%%%%%%%%%%%%%%%%%%%%%%%%%
%%%%%%%%%%%%%%%%%%%%%%%%%%%%%%%%%%%
%
%
%
\subsection{Initial Value Problem}
\begin{equation}
y^{\prime}(t) = \frac{2}{t} y + t^2e^t, 1\leq t \leq 2, y(1) = 0.
\label{eq:IVP}
\end{equation}
The numerical solution is compared with the given exact solution in Fig. \ref{fig:IVP}.

\begin{figure}
     \centering
    		 {\includestandalone[scale=1]{math609_pa5_comp_example_1_1}}
     \caption{Numerical and exact solutions of the IVP in Eq. \ref{eq:IVP} }
     \label{fig:IVP}
\end{figure}
%
%
%%%%%%%%%%%%%%%%%%%%%%%%%%%%%%%%%%
%%%%%%%%%%%%%%%%%%%%%%%%%%%%%%%%%%
%
%
\subsection{Current in an RLC Circuit}
The numerical solution of the current in the circuit is plotted in Fig. \ref{fig:RLC}. The first picture shows the transient nature of the current whereas in the second, the current approaches a stable value.
%\begin{equation}
\begin{align*} 
i^{\prime}(t) = C E^{\prime\prime}(t) + \frac{1}{R} E^{\prime}(t) + \frac{1}{L} E(t), i(0) = 0.\\
E(t) = e^{-.06\pi t}\times \sin{(2t-\pi)}, 0 < t < 5.
\label{eq:RLC}
\end{align*} 

%\end{equation}

\begin{figure}
     \centering
    		 \subfloat[][]{ \includestandalone[scale=1]{math609_pa5_comp_example_2_5}\label{tb_5}}\\
		 \subfloat[][]{ \includestandalone[scale=1]{math609_pa5_comp_example_2_50}\label{tb_50}}
     \caption{Current in an RLC Circuit (a) $t = 5$ (b) $t = 50$ sec}
     \label{fig:RLC}
\end{figure}


%%%%%%%%%%%%%%%%%%%%%%%%%%%%%%%
%%%%%%%%%%%%%%%%%%%%%%%%%%%%%%%
\subsection{Water flow in an inverted conical tank}

The water level inside the inverted conical tank is shown in Fig. \ref{fig:tank}. The tank will be emptied in around 1506 seconds according to the given conditions.

\begin{align*} 
y^{\prime}(t) = .6\pi r^2 \sqrt{2g}\frac{\sqrt{y}}{A(y)}, y(0) = 8\\
A(y) = \pi y^2.
\end{align*} 

\begin{figure}
     \centering
    		{ \includestandalone[scale=1]{math609_pa5_comp_example_3_1}\label{n_11}}
     \caption{Water level in an inverted conical tank}
     \label{fig:tank}
\end{figure}
%
%
%%%%%%%%%%%%%%%%%%%%%%%%%%%%%%%
%%%%%%%%%%%%%%%%%%%%%%%%%%%%%%%
\subsection{Potassium Hydroxide Stoichiometric Equation}

The number of potassium hoydroxide formed is shown in Fig. \ref{fig:CHEM}.
\begin{equation} 
\ch{2 K2Cr2O7 + 2 H2O + 3 S -> 4 KOH + 2 Cr2O3 + 3 SO2}
%\label{eq:Tank}
\end{equation} 


\begin{figure}
     \centering
    		 { \includestandalone[scale=1]{math609_pa5_comp_example_4_100}\label{n_11}}
     \caption{Number of $KOH$ molecules formed}
     \label{fig:CHEM}
\end{figure}
%
%


%%%%%%%%%%%%%%%%%%%%%%%%%%%%%%%
%%%%%%%%%%%%%%%%%%%%%%%%%%%%%%%

\end{document}








